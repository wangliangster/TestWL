%!TEX TS-program = xelatex
\documentclass[letterpaper,10pt]{Resume_Liang}

%set line space for Chinese and English share
\usepackage[nodisplayskipstretch]{setspace} % In my case
\setstretch{0.8} % 1.5/1.2 = 1.25
\usepackage{caption}
\captionsetup{font={stretch=\setstretch}}

\begin{document}
\thispagestyle{empty}
\header{王}{良} {目标岗:NLP程序员 / 机器学习/ 深度学习/ 数学研究员}
% In the aside, each new line forces a line break
\begin{aside}
  \section{Contact:}
    (+86)13810716443\\
    \href{mailto:mathmad@163.com}{mathmad@163.com}\\
    \href{mailto:wangliangster@gmail.com}{wangliang@gmail.com}\\
    北京大兴\\
   \vspace{\baselineskip}
  \section{Interests:}
  unsupervised learning\\
  \smallitem{+ clustering algorithms}\\
  \smallitem{+ dimensionality reduction}\\
   text analysis\\
  \smallitem{+ entity resolution}\\
  \smallitem{+ distributional semantics}\\
   harmonic analysis\\
  \smallitem{+Fourier transfer}\\
  \smallitem{+Wavelet analysis}\\
  Riemann Hypothesis\\
   \vspace{\baselineskip}
  \section{Algorithms:}
  \smallitem{PCA/SVD/SGD}\\
  \smallitem{CNN/RNN/LSTM}\\
  \smallitem{DFS/BFS/TF-IDF}\\
  \smallitem{NB/LR/SVM}\\
  \smallitem{KNN/XGBoost/LDA}\\
  \smallitem{Word2Vec/Doc2Vec}\\
  \smallitem{DFT/FFT}\\
 Tensorflow/Keras/Pytorch\\
  \vspace{\baselineskip}
  \section{Languages:}
    Python\\
    \smallitem{+ numpy/scipy/pandas}\\
    \smallitem{+ scikit-learn}\\
    \smallitem{+ jieba/ltp}\\
    \smallitem{+ gensim/word2vec}\\
    \smallitem{+ matplotlib}\\
    \smallitem{+ PyCharm}\\
    \smallitem{+ IPython}\\
    C/C++\\
    \smallitem{+ makefile/imake}\\
    \smallitem{+ Serial port, CAN port}\\
    \smallitem{+ CORBA(TAO)}\\
    Jupyter notebook\\
    JSON\\
    Tk/Tcl\\
    Perl\\
    Qt\\
    SQL\\
    Java\\
    \LaTeX \\
   \vspace{\baselineskip}
  \section{Computing:}
   Linux\\
   \smallitem{+ sed/awk/grep/vi/grub}\\
   \smallitem{+ SystemRescueCd}\\
   \smallitem{+ minicom/vnc/ftp/ssh}\\
   \smallitem{+ Anaconda}\\
   Common\\
   \smallitem{+ jekins}\\
   \smallitem{+ git}\\
   \smallitem{+ clearcase}\\
   \vspace{\baselineskip}

\end{aside}
\begin{body}
\section{工作经历}
\begin{entrylist}
  \entry
    {10/2018-now}
    {Artificial-Intelligence-for-NLP China Team}
    {算法工程师}
    {\href{https://github.com/Artificial-Intelligence-for-NLP-and-CV/comment-setimental-classification}{细粒度情感言论分类},\href{https://github.com/wangliangster/TestWL/blob/master/ViewPointExtract.ipynb}{新闻观点抽取}, \href{https://github.com/wangliangster/NLP-Course/blob/master/BeijingSubway.ipynb}{北京地铁换乘方案}, 运用了DFS/BFS图搜索, TF-IDF, 常用深度学习算法,以及生成 \href{https://github.com/wangliangster/TestWL/blob/master/wordcloudLiang.ipynb}{词云}等技术 }	
  \entry
    {5/2013-10/2018}
    {通用电气(GE)$\bullet$医疗集团}
    {高级软件工程师}
    {图像处理和自动化测试工具CI及OEM等项目开发Leader; \href{https://www.gehealthcare.com/en/products/radiography/fixed-rad-systems/discovery-xr656-hd-x-ray-system-powered-by-helix}{Discovery XR656 HD}, \href{https://www.gehealthcare.com/en/products/radiography/fixed-rad-systems/optima-xr646-hd-x-ray-system-powered-by-helix}{Optima XR646 HD}, \href{https://www.gehealthcare.com/en/products/radiography/fixed-rad-systems/discovery-xr656-plus}{Discovery XR656 plus}, \href{https://www.gehealthcare.com/en/products/radiography/fixed-rad-systems/optima-xr646}{Optima XR646} 三代X光机系统项目集成Leader; X-Ray爆光控制平台开发; 以及Ghost, Linux 系统恢复, ISO 生成和 CD/DVD 制作等软件工具团队主要贡献者}
  \entry
    {4/2011-5/2013}
    {中核控制系统工程有限公司(中核集团)}
    {资深软件工程师}
    {Nicsys1000系统开发, 基于QT开发了大量核电站DCS系统的HMI, 搭建SVN服务, 带领团队参加QT大会作深入交流}
\end{entrylist}

\section{教育背景}
\begin{entrylist}
  \entry
    {2008-2011}
    {硕士(计算机)}
    {上海大学}
    {Graduate course sequences in Combinatorics/Graph Theory, Probability, plus Computer Engineering}
  \entry
    {2003-2007}
    {本科(应用数学)}
    {江西财经大学}
    {Pure mathematics concentration. Courses in Analysis, Algebra, Combinatorics, Probability}
   \entry {\href{https://wangliangster.github.io}{个人主页:}}
   {\href{https://wangliangster.github.io}{https://wangliangster.github.io}}{注意:上下文所有蓝色字体均为有效超链接,详情可点入浏览}{}
\end{entrylist}

\section{项目经验}
\begin{entrylist}
 \projectentry
  {01/2019-now}
  {\href{https://github.com/Artificial-Intelligence-for-NLP-and-CV/comment-setimental-classification}{评论细粒度分类}}
  {NLP Project}
  {对来源于大众点评的数据集,进行6大类,20小类打标,对于每个小类,都会有< 正面情感, 中性情感, 负面情感, 情感倾向未提及 > 这4个类别}
  \projectentry
    {10/2018-01/2019}
    {\href{https://github.com/wangliangster/TestWL/blob/master/ViewPointExtract.ipynb}{新闻人物言论提取}}
    {NLP Project}
    {Use word2vec and TF-IDF, probability Algorithms, this project aims to for any input News text, extract the main view point of each speaker, present it as format "who said what" }
  \projectentry
    {11/2018-12/2018}
    {\href{https://github.com/wangliangster/NLP-Course/blob/master/BeijingSubway.ipynb}{地铁换乘图搜索解决方案}}
    {Hobby Graph Search Project}
    {Use BFS/DFS/A* Algorithms, demo a Graph search solution, land on Beijing Subway transfer search}
    \projectentry
    {3/2017-5/2018}
    {Reli}
    {GE Project}
    {Reli是GE内部一可靠性自动化测试工具,旨在测试GE新一代X光机产品的可靠性。积累了大量数据,可自动分析log生成报告,依据可靠性理论和数据指导项目开发的关注和待改进点,快速响应客户问题。用python, tk/tcl, shell scripts, ssh, minicom等技术基于Linux平台开发, 引入了随机算法,与旧工具每天人工VNC录屏的方式相比, 极大的提高了覆盖率,获得良好赞誉}
     \projectentry
    {10/2013-9/2017}
    {\href{https://www.gehealthcare.com/en/products/radiography/fixed-rad-systems}{Everest G1/G2/G3}}
    {GE Project}
    {Everest G1/G2/G3 are a sequences of GE's Radiography Family products which developed by global team, as a Lead Software Integrator, I am along with them all the way  }
\end{entrylist}

\section{其它}
\begin{entrylist}
   \otherentry
   {2018}
   {\href{https://auth.geeksforgeeks.org/user/\%E7\%8E\%8B\%E8\%89\%AF/practice/}{Rank in Institute \#1}}
   {GeeksforGeeks}
    \otherentry
    {2017}
    {BUILDING ESSENTIAL LEADERSHIP SKILLS (BELS) }
    {Crotonville Leadership}
     \otherentry
    {2016}
    {\href{https://github.com/wangliangster/TestWL/blob/master/Fouries\%20Transfer\%20Notes3.pdf}{Liang Wang; Fourier series deep explanation }}
    {GE Internal Seminar }
     %\otherentry
    %{2015}
   % {PRESENTATION SKILLS AT GE}
    %{Crotonville Leadership}
    \otherentry
    {2010}
    {\href{https://ieeexplore.ieee.org/document/5564214}{Liang Wang; Songnian Yu; Feng Chen:“A new solution of node splitting to the R Tree algorithm”}}
    {IEEE}
\end{entrylist}
\end{body}
\end{document}
